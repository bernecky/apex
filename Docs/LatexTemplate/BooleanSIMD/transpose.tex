\section{Transpose}
\label{transpose}

Matrix transpose seems an unlikely candidate for
SIMD optimization, but there are at least two cases
where it has been applied in APL.

Our implementation of transpose for SHARP APL performed
an optimization, similar to the one we used in rotate, on 
all array types. A dyadic transpose that leaves one or
trailing axes in place moves those elements as a unit,
using rbemove or its equivalent, {\em e.g.}:

\medskip

{\apl T\qlarrow\02~2~3~3\qrho\qiota\036\\
~~0~~1~~2\\
~~3~~4~~5\\
~~6~~7~~8\\
\\
~~9~10~11\\
12~13~14\\
~15~16~17\\\
\\
~18~19~20\\
~21~22~23\\
~24~25~26\\
\\
~27~28~29\\
~30~31~32\\
~33~34~35\\}


{\apl 1~0~2~3\qtran\0T\\
~~0~~1~~2\\
~~3~~4~~5\\
~~6~~7~~8\\
~\\
~18~19~20\\
~21~22~23\\
~24~25~26\\
~\\
~\\
~~9~10~11\\
12~13~14\\
15~16~17\\\
\\
27~28~29\\
30~31~32\\
33~34~35\\}

\noindent
A more interesting transpose algorithm applies to 
Boolean arrays whose shape is a multiple of eight
on its last two axes. Monadic transpose 
({\em e.g.}, {\apl \qtran\010000 75000\qrho\00 1}), or a 
dyadic transpose that interchanges only the last two argument axes
({\em e.g.}, {\apl 0 2 1\qtran\03 10000 75000\qrho\00 1}), can exploit
an algorithmic kernel for transposing Boolean arrays of shape {\apl 8~8} 
that appears, at very least, in 
{\em Hacker's Delight}.\cite{Warren:2012:HD:2462741}
The algorithm performs 16 logical and shift operations on
a 64-bit value that contains the ravel of the array, and
produces the transposed value as its result. The 8x8 results of
applying this transpose kernel are copied into the final result of the
larger array transpose. Jay Foad's implementation of this
approach for Dyalog APL produced a speedup of about ten times,
for 10000x75000 Boolean arrays.\cite{JFoad:pc2016}.
Section~\ref{compressexpand} describes two vector instructions,
{\tt PDEP} and {\tt PEXT},
that appear in some recent machine architectures; {\tt PDEP} can
be used to implement the above transpose kernel.
A {\em perfect shuffle}, as its name suggests, splits an array into
two halves, then selects elements from alternate halves.
Foad noted that two {\tt PDEP} instructions and an {\tt OR} produce a
highly efficient perfect shuffle verb, modeled in APL as {\apl S},
and that composing {\apl S} three times produces the 
transpose kernel described above:

{\apl S\qlarrow\qlbrace\qomega\qlbr\qugrade\qlpar\qrho\qomega\qrpar\qrho\00~1\qrbr\qrbrace}\\

{\apl T\qlarrow\08~8\qrho\qiota\064\\

{\apl 8~8\qrho\0S~S~S~\qcomma\0T}\\
0~~8~16~24~32~40~48~56\\
1~~9~17~25~33~41~49~57\\
2~10~18~26~34~42~50~58\\
3~11~19~27~35~43~51~59\\
4~12~20~28~36~44~52~60\\
5~13~21~29~37~45~53~61\\
6~14~22~30~38~46~54~62\\
7~15~23~31~39~47~55~63\\
}

\noindent In general, any square array, {\apl T}, 
whose shape vector is all powers of two, can be transposed with this verb,
where {\apl POW} is the APL {\em power} conjunction.

{\apl TR\qlarrow\qlbrace\0S POW\qlpar\02\qlog\01\qrho\qrho\qomega\qrpar\qomega\qrbrace}

\noindent We plan to investigate SIMD algorithms
for efficient transposition of Boolean arrays of shapes other than
powers of two.

Foad pointed out that the IBM POWER architecture
includes the {\tt vgbbd} instruction, which can perform
two 8x8 Boolean transposes at once. 
He notes that its performance is less than that of straightforward 
C code, because execution time is dominated by
argument setup and result storing for the {\tt vgbbd} instruction.

