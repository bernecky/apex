\section{Matrix Operations}
\label{matrixoperations}

Most of the above optimizations are defined here on vectors, but with
a bit of work to handle shards, are
extendable to higher-rank arguments.\footnote{That 
bit of work is often somewhat tedious, and usually less elegant,
than handling the byte-aligned part of a matrix row.}
{\em Shards} are the two sub-byte fragments, possibly empty, of a 
Boolean vector or matrix row that starts and/or ends in mid-byte. 
{\em E.g.,} in the vector:

\medskip
{\apl 1~1~0~0,~0~0~0~0,~1~0~1~0,~1~1~1~0}\\

\noindent the vector {\apl 1~0~1} is a shard, because its 
elements start at a byte boundary, but end in mid-byte.
Similarly, the vector {\apl 0~1~1~1~0} is a shard, because 
it starts in mid-byte, and ends on a byte boundary.

Shards and partial-word results are usually dealt with one of
two ways: either part or all of result is precleared to zero, then 
computed shards are ORd into the result, or the indexed
assign of the shard is handled by code, such as {\em rbemove}, 
that is guaranteed not to overwrite any part of the result
outside the bounds of the shard.
In the case of multi-thread execution, care must, of course, be taken
to deal with false sharing among cache lines, and of preventing
one thread from doing destructive writes of a partial result 
computed by another thread.

We are not aware of any mainframe implementations of SIMD optimizations
for non-trailing axis reductions or scans on Booleans.
However, Roger Hui presented an algorithm for first-axis Boolean reductions on 
non-byte-aligned matrices at the 2008 Dyalog APL 
Conference.~\cite{RHui:bitarrays}.

\fixme{Need good example}

