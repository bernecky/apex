\section{Strength Reduction}
\label{strengthreduction}

Strength reduction has been a common compiler optimization
technique for decades. Strength reduction replaces
one operation by an equivalent one that is, typically, faster
to execute. For example, a multiply of an integer by a
power of two could be replaced with a left shift by the base-two logarithm of
the power of two, if the shift operation was 
faster than the multiply.~\cite{DFBacon:transformations2}

Strength reduction has been used in APL interpreters for
decades. In the case of Boolean arguments to verbs,
some common strength reductions are, where {\apl B1 and B2} are
conformable Boolean arrays:

\medskip
\begin{tabular}{|l|l|}
\hline
Unoptimized & Optimized\\
\hline
{\apl B1\qtimes\0B2} & {\apl B1\qand\0B2}\\
{\apl B1\qdstile\0B2} & {\apl B1\qand\0B2}\\
{\apl B1\qustile\0B2} & {\apl B1\qor\0B2}\\
{\apl B1\qstar\0B2} & {\apl B1\qge\0B2}\\
{\apl B1\qstile\0B2} & {\apl B1\qlt\0B2}\\
{\apl B1\qbang\0B2} & {\apl B1\qle\0B2}\\
{\apl \qtimes\0B2} & {\apl B2}\\
\hline
\end{tabular}
\medskip

\noindent The reduction of expensive integer (or real) operations
to fast Boolean operations provides a significant performance boost.
Furthermore, operating in the Boolean domain provides further
performance benefits, by eliminating many conditionals 
within the element-wise computation, and permitting word-at-a-time
SIMD execution, even in architectures without vector extensions.

