
% Prepared for the Dyalog '16 conference in Glasgow

%%\usepackage[latin1]{inputenc}
\usepackage[T1]{fontenc}
\usepackage{amsmath}
\usepackage{epsfig}
\usepackage{beamerthemesplit}
\usepackage{verbatim}
\usepackage{saxpsa}
\usepackage{textcomp}

\usepackage{pgfplots}
\pgfplotsset{compat=1.9,width=0.9\paperwidth}

\begin{document}

\def\ibmapl{$APL\backslash\0360$}

\newcommand{\fig}[2]{{%
\begin{figure}%
\includegraphics[angle=00,width=0.95\textwidth]{#1}%
\caption{#2}%
\label{#1}%
\end{figure}%
}}

\newcommand{\fixme}[1]{{ \color{red} \bf FIXME: #1}}
\newcommand{\sac}{SaC}

%\conferenceinfo{Dyalog '16}{October 9-13, 2016, Glasgow, Scotland}

\title{APL SIMD Boolean Array Algorithms}
\author{Robert Bernecky}

\institute{Snake Island Research Inc, Canada\\
bernecky@snakeisland.com}

\maketitle

\frame{
\frametitle{Abstract}
\begin{abstract}
Computation on large Boolean arrays is becoming more prevalent,
due to applications such as cryptography, data compression, and image
analysis and synthesis.
The advent of bit-oriented vector
extensions for microprocessors and of GPUS presents opportunities for 
significant performance improvements in such Boolean-dominated applications.
Since APL is one of the few computer languages that supports 
dense (one bit per element, eight bits per byte), multi-dimensional Boolean 
arrays as first-class objects, it has naturally attracted
research into optimizations for improved performance of
Boolean array operations.
This paper presents some of the 
Single Instruction, Multiple Data (SIMD) Boolean-related optimizations that have
appeared in APL implementations, and suggests ways in which those optimizations
might be exploited using contemporary hardware.

\end{abstract}

}


\frame{
\frametitle{A BIT of Introduction}
\begin{itemize}
\item<+-| alert@+>  The bit: the fundamental unit of digital computing
\item<+-| alert@+>  Yet, few computer languages treat bits as basic data types
\item<+-| alert@+>  Fewer support multi-dimensional bit arrays (8 bits/byte)
\item<+-| alert@+>  Fewer yet provide array operations on Boolean arrays
\item<+-| alert@+>  Boolean arrays appear in image analysis, cryptography, data compression\dots
\item<+-| alert@+>  The burden of bit twiddling is left to the programmer
\item<+-| alert@+>  APL, however, simply treats Booleans as the integers 0 and 1
\item<+-| alert@+>  Boolean arrays are grist to APL's data-parallel, expressive mill!
\end{itemize}
}

\frame{
\frametitle{Why Does APL have One-bit Booleans?}
\begin{itemize}
\item<+-| alert@+> Blame Larry Breed: while designing \ibmapl,
\item<+-| alert@+> Breed decided to store Booleans densely, eight bits/byte
\item<+-| alert@+> Booleans were stored in row-major order, as are other arrays
\item<+-| alert@+> This eased indexing, structural and selection verbs, etc.
\item<+-| alert@+> Single-bit indexing was more expensive than word indexing\dots
\item<+-| alert@+> But it opened the door to SIMD Boolean array optimizations
\item<+-| alert@+> Those optimizations are the subject of this talk
\item<+-| alert@+> Speedups were usually 8X or 32X, but sometimes even more
\item<+-| alert@+> A half century later, Breed's decision remains brilliant
\item<+-| alert@+> These optimizations are still important and relevant
\item<+-| alert@+> GPU and SIMD vector facilities can exploit them
\end{itemize}
}

\frame{
\frametitle{Scalar Verbs}
\begin{itemize}
\item<+-| alert@+>Breed optimized many rank-0 (scalar) 
Boolean verbs {\em e.g.} 
\item<+-| alert@+> Boolean verbs: {\apl \qand,\qor,\qtilde,\qnand,\qnor\dots}
\item<+-| alert@+> Relational verbs: {\apl \qlt,\qle,\qeq,\qge,\qgt,\qne}
\item<+-| alert@+> SIMD application, a word at a time (32 bits on S/360)
\item<+-| alert@+> One or more of us optimized scalar extension, {\em e.g.}
\item<+-| alert@+> {\apl 1\qand\0B} would produce {\apl B},
\item<+-| alert@+> without doing {\em any} element-wise computations
\end{itemize}
}

\frame{
\frametitle{Strength Reduction: a Classic Compiler Optimization}
\begin{itemize}
\item<+-| alert@+> {\em Strength reduction}: replace one operation by a cheaper one
\item<+-| alert@+> {\em E.g.}, replace multiply by a power of two with a shift
\item<+-| alert@+> In APL, Boolean {\apl B1\qtimes\0B2} becomes {\apl B1\qand\0B2}
\item<+-| alert@+> In APL, Boolean {\apl B1\qdstile\0B2} becomes {\apl B1\qand\0B2}
\item<+-| alert@+> In APL, Boolean {\apl B1\qstar\0B2} becomes {\apl B1\qge\0B2}
\item<+-| alert@+> Performance boosts: simpler verbs, SIMD operation, no conditionals, stay in Boolean domain
\item<+-| alert@+> Performance boosts: In a compiler, opportunity for other optimizations
\end{itemize}
}

\frame{
\frametitle{Structural and Selection Verbs I}
\begin{itemize}
\item<+-| alert@+> catenate, laminate, rotate, reverse, rank, from,
\item<+-| alert@+> merge, take, drop\dots
\item<+-| alert@+> These verbs, {\em e.g.}, {\apl 1 0 1, 0 1 1 0 } 
have to handle array indices that are not byte-aligned
\item<+-| alert@+> We would like these to run SIMD, word-at-a-time, on Booleans
\item<+-| alert@+> We introduced rbemove: generalized stride-1 (ravel order) copier verb
\item<+-| alert@+>  {\apl snk\qlbr\0sni\qplus\qiota\0k\qrbr\qlarrow\0src\qlbr\0sri\qplus\qiota\0k\qrbr}
\item<+-| alert@+> Does not corrupt out-of-bounds array elements
\item<+-| alert@+> Operates in SIMD mode(s) whenever possible
\item<+-| alert@+> Supports all type conversions
\end{itemize}
}

\frame{
\frametitle{Structural and Selection Verbs II}
\begin{itemize}
\item<+-| alert@+> Operation on non-trailing array axes: 
\item<+-| alert@+> SIMD copy entire subarrays at once, {\em e.g.}\\
{\apl 1\qrotf\02~3~4\qrho\qiota\024}
\item<+-| alert@+> rbemove will copy 12 adjacent array elements at once
\end{itemize}
}


\frame{
\frametitle{Structural and Selection Verbs III}
\begin{itemize}
\item<+-| alert@+> {\apl 2~3~4\qrho\qiota\024\\
~0~~1~~2~~3\\
~4~~5~~6~~7\\
~8~~9~10~11\\
~\\
12~13~14~15\\
16~17~18~19\\
20~21~22~23\\}
\item<+-| alert@+> {\apl 1\qrotf\02~3~4\qrho\qiota\024\\
12~13~14~15\\
16~17~18~19\\
20~21~22~23\\
~\\
~0~~1~~2~~3\\
~4~~5~~6~~7\\
~8~~9~10~11\\}
\end{itemize}
}

\frame{
\frametitle{Reverse and Rotate on Booleans}
\begin{itemize}
\item<+-| alert@+> Bernecky, 1979: fast algorithms for {\apl \qrotate\0\qomega} \& {\apl \qalpha\qrotate\qomega}
\item<+-| alert@+> last-axis Boolean {\apl \qrotate\qomega} did a byte at a time, w/table lookup\dots\\
{\apl~~~~~~ RevTab\qlbr\0uint8~\qomega\qrbr}
\item<+-| alert@+> then byte-aligned the resulting vector, SIMD, a word at a time
\item<+-| alert@+> All non-last-axis operations copied entire cells at once, using rbemove
\end{itemize}
}

\frame{
\frametitle{Reverse and Rotate Performance on Booleans}
\fig{matrix_rotate_first_axis.jpg}{Rotate Rewritten: 50 milliunits = 2.85msec}
}

\frame{
\frametitle{Reshape}
\begin{itemize}
\item<+-| alert@+> Reshape allows element reuse, {\em e.g.}:\\
{~~~~~~\apl 8\qrho\01~0~0}\\
{\apl 1~0~0~1~0~0~1~0}
\item<+-| alert@+> Breed optimized Boolean reshape this way:
\item<+-| alert@+> Copy the argument to the result
\item<+-| alert@+> Catenate the partial result to itself, doubling its length,
until its tail is byte-aligned.
\item<+-| alert@+> Do an overlapped move, or ``smear" of the result to its tail
\end{itemize}
}

\frame{
\frametitle{Transpose I}
\begin{itemize}
\item<+-| alert@+> An unlikely candidate for SIMD, it would seem\dots
\item<+-| alert@+> {\apl \qalpha\qtran\qomega} with unchanged trailing axes
\item<+-| alert@+> {\apl T\qlarrow\02~2~2~3\qrho\qiota\024\\
~~0~~1~~2\\
~~3~~4~~5\\
~\\
~~6~~7~~8\\
~~9~10~11\\
~\\
~\\
~12~13~14\\
~15~16~17\\
~\\
~18~19~20\\
~21~22~23\\
}
\end{itemize}
}


\frame{
\frametitle{Transpose II}
\begin{itemize}
\item<+-| alert@+> Transpose with unchanged trailing axes
\item<+-| alert@+> SIMD copy six elements at once (rbemove)
\item<+-| alert@+> {\apl 1~0~2~3\qtran\0T\\
~~0~~1~~2\\
~~3~~4~~5\\
~\\
~12~13~14\\
~15~16~17\\
~\\
~\\
~~6~~7~~8\\
~~9~10~11\\
~\\
~18~19~20\\
~21~22~23\\
}
\end{itemize}
}

\frame{
\frametitle{Boolean Transpose III}
\begin{itemize}
\item<+-| alert@+> {\em Hacker's Delight}: fast 8x8 Boolean matrix transpose
\item<+-| alert@+> Kernel: 16 logical \& shift operations on 64-bit ravel
\item<+-| alert@+> Uses {\em perfect shuffle} (PDEP) on any power of two shape
\item<+-| alert@+> Dyalog APL (Foad): 10X speedup on large Boolean array transpose
\item<+-| alert@+> Kernel generalizes to any power of two, {\em e.g.}, 16x16, 32x32
\end{itemize}
}

\frame{
\frametitle{Search Verbs}
\begin{itemize}
\item<+-| alert@+> Bernecky, 1971: fast {\em indexof} and {\em set membership}
\item<+-| alert@+> All data types except reals with {\apl \qQuad\0ct\qne 0}
\item<+-| alert@+> {\apl \qlpar\qalpha\qiota\00 1\qrpar\qlbr\qomega\qrbr}
\item<+-| alert@+> {\apl \qlpar\qalpha\qiota\0\qQuad\0av\qrpar\qlbr\qomega\qrbr}
\item<+-| alert@+> Booleans: Vector search for first byte of interest
\item<+-| alert@+> Then, table lookup to get bit offset
\item<+-| alert@+> Speedup: lots - linear time {\em vs.} quadratic time
\item<+-| alert@+> Created indexof kernel utility for interpreter use
\end{itemize}
}

\frame{
\frametitle{Reduce}
\begin{itemize}
\item<+-| alert@+> Roger D. Moore, 1971: fast {\apl \qplus\qslash\qomega} Boolean vector
\item<+-| alert@+> Initial use was {\em compress} and {\em expand} setup:\\
{\apl \qplus\qslash\qalpha} was taking longer than {\em compress}/{\em expand}
\item<+-| alert@+> S/360 translate vector op: Boolean bytes into population counts, SIMD 124 bytes per segment
\item<+-| alert@+> SIMD integer sum of 4-byte words gave 4-element partial sum
\item<+-| alert@+> Shift-and-add gave final result
\item<+-| alert@+> Segment size limited to prevent inter-byte carries
\item<+-| alert@+> Larry Breed haiku: {\apl \qplus\qslash\qplus\qslashf\04~resh~PopcountTab\qlbr\0uint8~\qomega\qrbr}
\item<+-| alert@+> Algorithm used briefly for {\apl \qor\qslash\qomega} and
{\apl \qand\qslash\qomega}
\end{itemize}
}

\frame{
\frametitle{Reduce and Scan}
\begin{itemize}
\item<+-| alert@+> E.E. McDonnell, 1974: elegant APL models of Boolean {\em scan} and {\em reduction} for relationals
\item<+-| alert@+>  Result was catenation of prefix, infix, \& suffix expressions 
\item<+-| alert@+> Used Bernecky's fast indexof
\item<+-| alert@+> Result: linear-time, word-at-a-time, SIMD Boolean scan \& reduce
\end{itemize}
}

\frame{
\frametitle{Scan}
\begin{itemize}
\item<+-| alert@+> John Heckman, 1970 or 1971: user-defined APL scan verbs
\item<+-| alert@+> Now widely used in GPUs
\item<+-| alert@+> Recursive doubling:\\
{\apl
r\qlarrow\0nescanall y\qsemic\0s\qsemic\0biw\\
~~\qlamp~Not-equal scan\\
~~r\qlarrow\0y\\
~~biw\qlarrow\qustile\02\qlog\01\qustile\qrho\0y\\
~~\qcolon\0For s \qcolon\0In 2\qstar\qiota\0biw \qlamp~Heckman\\
~~~~r\qlarrow\0←r\qne\qlpar\qbar\qrho\0r\qrpar\quarrow\qlpar\qbar\0s\qrpar\qdarrow\0r\\
~~\qcolon\0EndFor\\
}
\item<+-| alert@+> SIMD, word-at-a-time algorithm for Boolean {\apl \qne\qbslash\qomega} and {\apl\qeq\qbslash\qomega} along last axis
\item<+-| alert@+> Bernecky's simple C Heckman implementation is about 3X faster than Dyalog APL 15.0 (vector only)
\item<+-| alert@+> So far, no X86 vectorization; perhaps we can do even better
\end{itemize}
}

\frame{
\frametitle{STAR Inner Product I}
\begin{itemize}
\item<+-| alert@+> IPSA, 1973: Boolean array inner products were painfully slow
\item<+-| alert@+> Control Data (CDC) wanted APL for their new STAR-100 vector supercomputer
\item<+-| alert@+> Group from Toronto I.P. Sharp Associates hired to work on the interpreter
\item<+-| alert@+> Memory-to-memory vector instructions needed stride-1 access for good performance
\item<+-| alert@+> Bernecky: heard about STAR APL stride-1 inner-product algorithm;
\item<+-| alert@+> redesigned Boolean inner product to use STAR algorithm
\end{itemize}
}

\frame{
\frametitle{Classic Inner Product Algorithm}
{\apl

\qZ\0\qlarrow\0\qX\0\qblank\0\qi\0\qp\0\qc\0\ql\0\qa\0\qs\0\qs\0\qi\0\qc\0\qblank\0\qY\0\qsemic\0\qR\0\qX\0\qsemic\0\qC\0\qX\0\qsemic\0\qC\0\qY\0\qsemic\0\qI\0\qsemic\0\qJ\0\qsemic\0\qK\0\qblank\0\qblank\0\qblank\0\qblank\0\qblank\0\qblank\0\qblank\0\qblank\0\qblank\0\\
\qblank\0\qR\0\qX\0\qlarrow\0\qlpar\0\qrho\0\qX\0\qrpar\0\qlbr\00\qrbr\0\qblank\0\qblank\0\qblank\0\qblank\0\qblank\0\qblank\0\qblank\0\qblank\0\qblank\0\qblank\0\qblank\0\qblank\0\qblank\0\qblank\0\qblank\0\qblank\0\qblank\0\qblank\0\qblank\0\qblank\0\qblank\0\qblank\0\qblank\0\qblank\0\qblank\0\qblank\0\qblank\0\qblank\0\qblank\0\\
\qblank\0\qC\0\qX\0\qlarrow\0\qlpar\0\qrho\0\qX\0\qrpar\0\qlbr\01\qrbr\0\qblank\0\qblank\0\qblank\0\qblank\0\qblank\0\qblank\0\qblank\0\qblank\0\qblank\0\qblank\0\qblank\0\qblank\0\qblank\0\qblank\0\qblank\0\qblank\0\qblank\0\qblank\0\qblank\0\qblank\0\qblank\0\qblank\0\qblank\0\qblank\0\qblank\0\qblank\0\qblank\0\qblank\0\qblank\0\\
\qblank\0\qC\0\qY\0\qlarrow\0\qlpar\0\qrho\0\qY\0\qrpar\0\qlbr\01\qrbr\0\qblank\0\qblank\0\qblank\0\qblank\0\qblank\0\qblank\0\qblank\0\qblank\0\qblank\0\qblank\0\qblank\0\qblank\0\qblank\0\qblank\0\qblank\0\qblank\0\qblank\0\qblank\0\qblank\0\qblank\0\qblank\0\qblank\0\qblank\0\qblank\0\qblank\0\qblank\0\qblank\0\qblank\0\qblank\0\\
\qblank\0\qZ\0\qlarrow\0\qlpar\0\qR\0\qX\0\qcomma\0\qC\0\qY\0\qrpar\0\qrho\00\qdot\05\qblank\0\qblank\0\qblank\0\qblank\0\qblank\0\qblank\0\qblank\0\qblank\0\qblank\0\qblank\0\qblank\0\qblank\0\qblank\0\qblank\0\qblank\0\qblank\0\qblank\0\qblank\0\qblank\0\qblank\0\qblank\0\qblank\0\qblank\0\qblank\0\qblank\0\qblank\0\\
\qblank\0\qcolon\0\qF\0\qo\0\qr\0\qblank\0\qI\0\qblank\0\qcolon\0\qI\0\qn\0\qblank\0\qiota\0\qR\0\qX\0\qblank\0\qblank\0\qblank\0\qblank\0\qblank\0\qblank\0\qblank\0\qblank\0\qblank\0\qblank\0\qblank\0\qblank\0\qblank\0\qblank\0\qblank\0\qblank\0\qblank\0\qblank\0\qblank\0\qblank\0\qblank\0\qblank\0\qblank\0\qblank\0\qblank\0\\
\qblank\0\qblank\0\qcolon\0\qF\0\qo\0\qr\0\qblank\0\qJ\0\qblank\0\qcolon\0\qI\0\qn\0\qblank\0\qiota\0\qC\0\qY\0\qblank\0\qblank\0\qblank\0\qblank\0\qblank\0\qblank\0\qblank\0\qblank\0\qblank\0\qblank\0\qblank\0\qblank\0\qblank\0\qblank\0\qblank\0\qblank\0\qblank\0\qblank\0\qblank\0\qblank\0\qblank\0\\
\qblank\0\qblank\0\qblank\0\qZ\0\qlbr\0\qI\0\qsemic\0\qJ\0\qrbr\0\qlarrow\00\qblank\0\qblank\0\qblank\0\qblank\0\qblank\0\qblank\0\qblank\0\qblank\0\qblank\0\qblank\0\qblank\0\qblank\0\qblank\0\qblank\0\qblank\0\qblank\0\qblank\0\qblank\0\qblank\0\qblank\0\qblank\0\qblank\0\qblank\0\\
\qblank\0\qblank\0\qblank\0\qcolon\0\qF\0\qo\0\qr\0\qblank\0\qK\0\qblank\0\qcolon\0\qI\0\qn\0\qblank\0\qiota\0\qC\0\qX\0\qblank\0\qblank\0\qblank\0\qblank\0\qblank\0\qblank\0\qblank\0\qblank\0\qblank\0\qblank\0\qblank\0\qblank\0\qblank\0\qblank\0\qblank\0\qblank\0\qblank\0\\
\qblank\0\qblank\0\qblank\0\qblank\0\qZ\0\qlbr\0\qI\0\qsemic\0\qJ\0\qrbr\0\qlarrow\0\qZ\0\qlbr\0\qI\0\qsemic\0\qJ\0\qrbr\0\qplus\0\qX\0\qlbr\0\qI\0\qsemic\0\qK\0\qrbr\0\qtimes\0\qY\0\qlbr\0\qK\0\qsemic\0\qJ\0\qrbr\0\\
\qblank\0\qblank\0\qblank\0\qcolon\0\qE\0\qn\0\qd\0\qF\0\qo\0\qr\0\qblank\0\qblank\0\qblank\0\qblank\0\qblank\0\qblank\0\qblank\0\qblank\0\qblank\0\qblank\0\qblank\0\qblank\0\qblank\0\qblank\0\qblank\0\qblank\0\qblank\0\qblank\0\qblank\0\qblank\0\qblank\0\qblank\0\qblank\0\qblank\0\\
\qblank\0\qblank\0\qcolon\0\qE\0\qn\0\qd\0\qF\0\qo\0\qr\0\qblank\0\qblank\0\qblank\0\qblank\0\qblank\0\qblank\0\qblank\0\qblank\0\qblank\0\qblank\0\qblank\0\qblank\0\qblank\0\qblank\0\qblank\0\qblank\0\qblank\0\qblank\0\qblank\0\qblank\0\qblank\0\qblank\0\qblank\0\qblank\0\qblank\0\qblank\0\qblank\0\qblank\0\\
\qblank\0\qcolon\0\qE\0\qn\0\qd\0\qF\0\qo\0\qr\0\qblank\0\qblank\0\qblank\0\qblank\0\qblank\0\qblank\0\qblank\0\qblank\0\qblank\0\qblank\0\qblank\0\qblank\0\qblank\0\qblank\0\qblank\0\qblank\0\qblank\0\qblank\0\qblank\0\qblank\0\qblank\0\qblank\0\qblank\0\qblank\0\qblank\0\qblank\0\qblank\0\qblank\0\qblank\0\qblank\0\qblank\0\qblank\0\\
}

}

\frame{
\frametitle{STAR Inner Product Algorithm}
\input{ipstar}
}


\frame{
\frametitle{STAR Inner Product II}
\begin{itemize}
\item<+-| alert@+> Inner product loops reordered; key benefits, for 
{\apl \qalpha\0 f\qdot\0g \qomega}
\item<+-| alert@+> Each {\apl \qalpha} element, {\apl Xel}, fetched only once
\item<+-| alert@+> Type conversion of {\apl Xel} no longer time-critical
\item<+-| alert@+> {\apl Xel} analysis amortized over entire row: {\apl Y\qlbr\0J\qsemic\qrbr}
\item<+-| alert@+> Scalar-vector application of {\apl g}\\
{\apl tmp\qlarrow\0Xel g Y\qlbr\0J\qsemic\qrbr}
\item<+-| alert@+> Vector-vector {\apl f}-reduce into result row {\apl Z\qlbr\0I\qsemic\qrbr}\\
{\apl Z\qlbr\0I\qsemic\qrbr\qlarrow\0Z\qlbr\0I\qsemic\qrbr\qblank\0f tmp}
\end{itemize}
}

\frame{
\frametitle{SIMD Boolean STAR Inner Product Basics}
\begin{itemize}
\item<+-| alert@+> Scalar-vector {\apl Xel g Y\qlbr\0J\qsemic\qrbr} is word-at-a-time Boolean SIMD
\item<+-| alert@+> Vector-vector {\apl Z\qlbr\0I\qsemic\qrbr\qlarrow\0Z\qlbr\0I\qsemic\qrbr} is word-at-a-time Boolean SIMD
\item<+-| alert@+> We are already a lot faster\!
\item<+-| alert@+> The STAR APL model does {\apl \qplus\qdot\qtimes} 90X faster than the Classic model, on 200x200 real matrices
\item<+-| alert@+> Unfortunately, the APL primitive is still 30X faster than the APL model
\end{itemize}
}

\frame{
\frametitle{Boolean STAR Inner Product Optimizations}
\begin{itemize}
\item<+-| alert@+> Consider {\apl tmp\qlarrow\0Xel \qand\qblank\0RO} in {\apl \qalpha \qor\qdot\qand \qomega}
\item<+-| alert@+> If {\apl Xel} is {\apl 0}, then {\apl tmp} is all zeros: no {\apl g} computation
\item<+-| alert@+> If {\apl Xel} is {\apl 1}, then {\apl tmp} is just {\apl RO}:
no {\apl g} computation
\item<+-| alert@+> {\apl f} is {\apl \qor}, so its identity element is {\apl 0}
\item<+-| alert@+> Hence, if {\apl Xel} is {\apl 0}, we can skip the {\apl g}-reduction
\item<+-| alert@+> Similarly, if {\apl Xel} is {\apl 1}, we can do the {\apl g}-reduction using {\apl RO}
\item<+-| alert@+> This gives us poor man's sparse arrays, which works on other data types, too
\item<+-| alert@+> Final result: Boolean inner products on SHARP APL/PC ran
much faster than APL2 on huge mainframe
\end{itemize}
}

\frame{
\frametitle{Control Flow Becomes Data Flow}
\begin{itemize}
\item<+-| alert@+> Booleans as arithmetic: replace control flow by data flow
\item<+-| alert@+> Conditionals can often be removed, {\em e.g.}:
\item<+-| alert@+> Give those with salary, {\apl S}, less than {\apl Tiny} a raise of {\apl R}\\
{\apl S\qlarrow\0S\qplus\0R\qtimes\0S\qlt\0Tiny}
\item<+-| alert@+> Knuth calls this capability {\em Iverson's convention for
characteristic functions}
\item<+-| alert@+> See also the verb {\apl mqs}, for finding quoted text
\item<+-| alert@+> See also the Bernecky-Scholz PLDI2014 Arrays Workshop paper: {\em Abstract Expressionism}
\end{itemize}
}

\frame{
\frametitle{Boolean Sort}
\begin{itemize}
\item<+-| alert@+> Sort ascending for Booleans:\\
{\apl SortAscending\qlarrow\qlbrace\qlpar\qbar\qrho\qomega\qrpar\quarrow\qlpar\qplus\qslash\qomega\qrpar\qrho\01\qrbrace}

\item<+-| alert@+> Boolean sort can use Moore's SIMD {\apl \qplus\qslash\0Boolean} in its first phase of execution
\item<+-| alert@+> Second phase can be performed in SIMD, {\em e.g.}, by a single SAC data-parallel with-loop.
\end{itemize}
}

\frame{
\frametitle{Boolean Grade}
\begin{itemize}
\item<+-| alert@+> SIMD Boolean upgrade:\\
{\apl ug\qlarrow\qlbrace\qlpar\qlpar\qtilde\qomega\qrpar\qslash\qiota\qrho\qomega\qrpar\qcomma\qomega\qslash\qiota\qrho\qomega\qrbrace}
\item<+-| alert@+> Not stunningly SIMD, though.
\end{itemize}
}

\frame{
\frametitle{Boolean Matrix Operations}
\begin{itemize}
\item<+-| alert@+> {\em Shard}: For byte-oriented algorithms,
a possibly empty sub-byte fragment of
a matrix row, extending from the start of the row to next byte,
or from the last byte in the row to the end of the row.
\item<+-| alert@+> Handling shards is a nuisance; it destroys algorithmic beauty
\item<+-| alert@+> Handling shards is also slower than beautiful algorithms
\item<+-| alert@+> Consider:\\ {\apl 1~1~0~0,~0~0~0~0,~1~0~1~0,~1~1~1~0}
\item<+-| alert@+> The vector {\apl 1~0~1} is a shard, because its
elements start at a byte boundary, but end in mid-byte.
\item<+-| alert@+> The vector {\apl 0~1~1~1~0} is a shard, because
it starts in mid-byte, and ends on a byte boundary.
\item<+-| alert@+> A similar definition holds for word-oriented algorithms
\end{itemize}
}

\frame{
\frametitle{Acknowledgements}
My sincere thanks to James A. Brown, Larry M. Breed, Walter Fil, 
Jay Foad, Roger K.W. Hui, Roger D. Moore, and Bob Smith,
for their constructive suggestions and ideas regarding this paper.
~\\
The APL programs in this paper were executed on Dyalog APL 
Version 15.0. Dyalog provides free downloads of their interpreters
for educational use; they also offer a free download of their
Raspberry Pi edition, at {\tt www.dyalog.com}.
~\\
The British APL Association (BAA) provided the author with
financial assistance for attending this conference;
their thoughtfulness and generosity is greatly appreciated.
}

