\section{Inner Product}
\label{innerproduct}

Iverson generalized linear algebra's inner 
product.~\cite{Iverson62,KEIverson:tot}
In his groundbreaking book, he said, in regard to
{\em generalized matrix product}:

\begin{quote}
This formulation emphasizes the fact that matrix multiplication
incorporates two elementary operations (${+}, {\times}$) and 
suggests that they be displayed explicitly. The ordinary
matrix product will therefore be written as: 
$X \stackrel{+}{\times} Y$.
\end{quote}

For \ibmapl, that notation had to be linearized to allow entry
and display on the typewriter-like terminals of that era. 
Hence, the ordinary
inner product is now written as: {\apl X\qplus\qdot\qtimes\0Y},
and the general case as: {\apl X f\qdot\0g Y}, where {\apl f}
and {\apl g} are any suitable verbs.
The generalized inner product has many uses: one step
of the Boolean-Boolean transitive-closure computation on a connectivity matrix
can be computed as: {\apl M\qor\qdot\qand\0M}.
A weaving on a loom can be modeled as a single inner product: a matrix,
{\apl tieup}, specifies which harness(es) should be raised
when a given treadle is pushed; {\apl tread} specifies the
treadling order; {\apl thread} specifies which threads are
in the weft:

\medskip
{\apl tieup\qlbr\0tread\qsemic\qrbr\qor\qdot\qeq\0thread}
\medskip

\fixme{Give example}

The classical algorithm for inner product is painfully slow for all
data types, but for Boolean arrays, it is excruciatingly so.
In the early 1970s, the author heard of an inner-product algorithm
that had been used by colleagues at I.P. Sharp Associates,
who were developing the STAR APL interpreter for the CDC STAR-100, 
a memory-to-memory supercomputer that required stride-1
algorithms for acceptable performance. The STAR APL algorithm 
worked by interchanging the loop ordering. The classic inner product
can be written as:

{\apl

\qZ\0\qlarrow\0\qX\0\qblank\0\qi\0\qp\0\qc\0\ql\0\qa\0\qs\0\qs\0\qi\0\qc\0\qblank\0\qY\0\qsemic\0\qR\0\qX\0\qsemic\0\qC\0\qX\0\qsemic\0\qC\0\qY\0\qsemic\0\qI\0\qsemic\0\qJ\0\qsemic\0\qK\0\qblank\0\qblank\0\qblank\0\qblank\0\qblank\0\qblank\0\qblank\0\qblank\0\qblank\0\\
\qblank\0\qR\0\qX\0\qlarrow\0\qlpar\0\qrho\0\qX\0\qrpar\0\qlbr\00\qrbr\0\qblank\0\qblank\0\qblank\0\qblank\0\qblank\0\qblank\0\qblank\0\qblank\0\qblank\0\qblank\0\qblank\0\qblank\0\qblank\0\qblank\0\qblank\0\qblank\0\qblank\0\qblank\0\qblank\0\qblank\0\qblank\0\qblank\0\qblank\0\qblank\0\qblank\0\qblank\0\qblank\0\qblank\0\qblank\0\\
\qblank\0\qC\0\qX\0\qlarrow\0\qlpar\0\qrho\0\qX\0\qrpar\0\qlbr\01\qrbr\0\qblank\0\qblank\0\qblank\0\qblank\0\qblank\0\qblank\0\qblank\0\qblank\0\qblank\0\qblank\0\qblank\0\qblank\0\qblank\0\qblank\0\qblank\0\qblank\0\qblank\0\qblank\0\qblank\0\qblank\0\qblank\0\qblank\0\qblank\0\qblank\0\qblank\0\qblank\0\qblank\0\qblank\0\qblank\0\\
\qblank\0\qC\0\qY\0\qlarrow\0\qlpar\0\qrho\0\qY\0\qrpar\0\qlbr\01\qrbr\0\qblank\0\qblank\0\qblank\0\qblank\0\qblank\0\qblank\0\qblank\0\qblank\0\qblank\0\qblank\0\qblank\0\qblank\0\qblank\0\qblank\0\qblank\0\qblank\0\qblank\0\qblank\0\qblank\0\qblank\0\qblank\0\qblank\0\qblank\0\qblank\0\qblank\0\qblank\0\qblank\0\qblank\0\qblank\0\\
\qblank\0\qZ\0\qlarrow\0\qlpar\0\qR\0\qX\0\qcomma\0\qC\0\qY\0\qrpar\0\qrho\00\qdot\05\qblank\0\qblank\0\qblank\0\qblank\0\qblank\0\qblank\0\qblank\0\qblank\0\qblank\0\qblank\0\qblank\0\qblank\0\qblank\0\qblank\0\qblank\0\qblank\0\qblank\0\qblank\0\qblank\0\qblank\0\qblank\0\qblank\0\qblank\0\qblank\0\qblank\0\qblank\0\\
\qblank\0\qcolon\0\qF\0\qo\0\qr\0\qblank\0\qI\0\qblank\0\qcolon\0\qI\0\qn\0\qblank\0\qiota\0\qR\0\qX\0\qblank\0\qblank\0\qblank\0\qblank\0\qblank\0\qblank\0\qblank\0\qblank\0\qblank\0\qblank\0\qblank\0\qblank\0\qblank\0\qblank\0\qblank\0\qblank\0\qblank\0\qblank\0\qblank\0\qblank\0\qblank\0\qblank\0\qblank\0\qblank\0\qblank\0\\
\qblank\0\qblank\0\qcolon\0\qF\0\qo\0\qr\0\qblank\0\qJ\0\qblank\0\qcolon\0\qI\0\qn\0\qblank\0\qiota\0\qC\0\qY\0\qblank\0\qblank\0\qblank\0\qblank\0\qblank\0\qblank\0\qblank\0\qblank\0\qblank\0\qblank\0\qblank\0\qblank\0\qblank\0\qblank\0\qblank\0\qblank\0\qblank\0\qblank\0\qblank\0\qblank\0\qblank\0\\
\qblank\0\qblank\0\qblank\0\qZ\0\qlbr\0\qI\0\qsemic\0\qJ\0\qrbr\0\qlarrow\00\qblank\0\qblank\0\qblank\0\qblank\0\qblank\0\qblank\0\qblank\0\qblank\0\qblank\0\qblank\0\qblank\0\qblank\0\qblank\0\qblank\0\qblank\0\qblank\0\qblank\0\qblank\0\qblank\0\qblank\0\qblank\0\qblank\0\qblank\0\\
\qblank\0\qblank\0\qblank\0\qcolon\0\qF\0\qo\0\qr\0\qblank\0\qK\0\qblank\0\qcolon\0\qI\0\qn\0\qblank\0\qiota\0\qC\0\qX\0\qblank\0\qblank\0\qblank\0\qblank\0\qblank\0\qblank\0\qblank\0\qblank\0\qblank\0\qblank\0\qblank\0\qblank\0\qblank\0\qblank\0\qblank\0\qblank\0\qblank\0\\
\qblank\0\qblank\0\qblank\0\qblank\0\qZ\0\qlbr\0\qI\0\qsemic\0\qJ\0\qrbr\0\qlarrow\0\qZ\0\qlbr\0\qI\0\qsemic\0\qJ\0\qrbr\0\qplus\0\qX\0\qlbr\0\qI\0\qsemic\0\qK\0\qrbr\0\qtimes\0\qY\0\qlbr\0\qK\0\qsemic\0\qJ\0\qrbr\0\\
\qblank\0\qblank\0\qblank\0\qcolon\0\qE\0\qn\0\qd\0\qF\0\qo\0\qr\0\qblank\0\qblank\0\qblank\0\qblank\0\qblank\0\qblank\0\qblank\0\qblank\0\qblank\0\qblank\0\qblank\0\qblank\0\qblank\0\qblank\0\qblank\0\qblank\0\qblank\0\qblank\0\qblank\0\qblank\0\qblank\0\qblank\0\qblank\0\qblank\0\\
\qblank\0\qblank\0\qcolon\0\qE\0\qn\0\qd\0\qF\0\qo\0\qr\0\qblank\0\qblank\0\qblank\0\qblank\0\qblank\0\qblank\0\qblank\0\qblank\0\qblank\0\qblank\0\qblank\0\qblank\0\qblank\0\qblank\0\qblank\0\qblank\0\qblank\0\qblank\0\qblank\0\qblank\0\qblank\0\qblank\0\qblank\0\qblank\0\qblank\0\qblank\0\qblank\0\qblank\0\\
\qblank\0\qcolon\0\qE\0\qn\0\qd\0\qF\0\qo\0\qr\0\qblank\0\qblank\0\qblank\0\qblank\0\qblank\0\qblank\0\qblank\0\qblank\0\qblank\0\qblank\0\qblank\0\qblank\0\qblank\0\qblank\0\qblank\0\qblank\0\qblank\0\qblank\0\qblank\0\qblank\0\qblank\0\qblank\0\qblank\0\qblank\0\qblank\0\qblank\0\qblank\0\qblank\0\qblank\0\qblank\0\qblank\0\qblank\0\\
}


\noindent By comparison, the STAR APL algorithm can be written
in APL this way, with one less level of visible loop 
nesting:\footnote{The mere removal of one level of explicit
looping makes the STAR APL model execute about 90 times
faster than the Classic model, on 200x200 matrices of
real numbers. Of course, that is still 30 times slower than
the APL derived function.}

{\apl

\qZ\0\qlarrow\0\qX\0\qblank\0\qi\0\qp\0\qs\0\qt\0\qa\0\qr\0\qblank\0\qY\0\qsemic\0\qR\0\qX\0\qsemic\0\qC\0\qX\0\qsemic\0\qC\0\qY\0\qsemic\0\qI\0\qsemic\0\qJ\0\qsemic\0\qX\0\qe\0\ql\0\qblank\0\qblank\0\qblank\0\qblank\0\\
\qblank\0\qR\0\qX\0\qlarrow\0\qlpar\0\qrho\0\qX\0\qrpar\0\qlbr\00\qrbr\0\qblank\0\qblank\0\qblank\0\qblank\0\qblank\0\qblank\0\qblank\0\qblank\0\qblank\0\qblank\0\qblank\0\qblank\0\qblank\0\qblank\0\qblank\0\qblank\0\qblank\0\qblank\0\qblank\0\qblank\0\qblank\0\qblank\0\qblank\0\\
\qblank\0\qC\0\qX\0\qlarrow\0\qlpar\0\qrho\0\qX\0\qrpar\0\qlbr\01\qrbr\0\qblank\0\qblank\0\qblank\0\qblank\0\qblank\0\qblank\0\qblank\0\qblank\0\qblank\0\qblank\0\qblank\0\qblank\0\qblank\0\qblank\0\qblank\0\qblank\0\qblank\0\qblank\0\qblank\0\qblank\0\qblank\0\qblank\0\qblank\0\\
\qblank\0\qC\0\qY\0\qlarrow\0\qlpar\0\qrho\0\qY\0\qrpar\0\qlbr\01\qrbr\0\qblank\0\qblank\0\qblank\0\qblank\0\qblank\0\qblank\0\qblank\0\qblank\0\qblank\0\qblank\0\qblank\0\qblank\0\qblank\0\qblank\0\qblank\0\qblank\0\qblank\0\qblank\0\qblank\0\qblank\0\qblank\0\qblank\0\qblank\0\\
\qblank\0\qZ\0\qlarrow\0\qlpar\0\qR\0\qX\0\qcomma\0\qC\0\qY\0\qrpar\0\qrho\00\qblank\0\qblank\0\qblank\0\qblank\0\qblank\0\qblank\0\qblank\0\qblank\0\qblank\0\qblank\0\qblank\0\qblank\0\qblank\0\qblank\0\qblank\0\qblank\0\qblank\0\qblank\0\qblank\0\qblank\0\qblank\0\qblank\0\\
\qblank\0\qcolon\0\qF\0\qo\0\qr\0\qblank\0\qI\0\qblank\0\qcolon\0\qI\0\qn\0\qblank\0\qiota\0\qR\0\qX\0\qblank\0\qblank\0\qblank\0\qblank\0\qblank\0\qblank\0\qblank\0\qblank\0\qblank\0\qblank\0\qblank\0\qblank\0\qblank\0\qblank\0\qblank\0\qblank\0\qblank\0\qblank\0\qblank\0\\
\qblank\0\qblank\0\qcolon\0\qF\0\qo\0\qr\0\qblank\0\qJ\0\qblank\0\qcolon\0\qI\0\qn\0\qblank\0\qiota\0\qC\0\qX\0\qblank\0\qblank\0\qblank\0\qblank\0\qblank\0\qblank\0\qblank\0\qblank\0\qblank\0\qblank\0\qblank\0\qblank\0\qblank\0\qblank\0\qblank\0\\
\qblank\0\qblank\0\qblank\0\qX\0\qe\0\ql\0\qlarrow\0\qX\0\qlbr\0\qI\0\qsemic\0\qJ\0\qrbr\0\qblank\0\qblank\0\qblank\0\qblank\0\qblank\0\qblank\0\qblank\0\qblank\0\qblank\0\qblank\0\qblank\0\qblank\0\qblank\0\qblank\0\qblank\0\\
\qblank\0\qblank\0\qblank\0\qZ\0\qlbr\0\qI\0\qsemic\0\qrbr\0\qlarrow\0\qZ\0\qlbr\0\qI\0\qsemic\0\qrbr\0\qplus\0\qX\0\qe\0\ql\0\qtimes\0\qY\0\qlbr\0\qJ\0\qsemic\0\qrbr\0\qblank\0\qblank\0\qblank\0\qblank\0\\
\qblank\0\qblank\0\qcolon\0\qE\0\qn\0\qd\0\qF\0\qo\0\qr\0\qblank\0\qblank\0\qblank\0\qblank\0\qblank\0\qblank\0\qblank\0\qblank\0\qblank\0\qblank\0\qblank\0\qblank\0\qblank\0\qblank\0\qblank\0\qblank\0\qblank\0\qblank\0\qblank\0\qblank\0\qblank\0\qblank\0\\
\qblank\0\qcolon\0\qE\0\qn\0\qd\0\qF\0\qo\0\qr\0\qblank\0\qblank\0\qblank\0\qblank\0\qblank\0\qblank\0\qblank\0\qblank\0\qblank\0\qblank\0\qblank\0\qblank\0\qblank\0\qblank\0\qblank\0\qblank\0\qblank\0\qblank\0\qblank\0\qblank\0\qblank\0\qblank\0\qblank\0\qblank\0\qblank\0\qblank\0\\
}



The STAR APL algorithm has a number of advantages over the classical
one: First, it references {\tt Xel}, an element of {\tt X}, only once, and
does so in stride-1 (ravel) order. Second, any type conversion or
analysis of {\apl Xel} is amortized over
the inner loop, so its cost is usually negligible. 
Third, the multiplies ({\apl g}) in the inner loop are stride-1, 
scalar-vector computations, so
they make good use of cache and are trivially vectorizable.
Fourth, the sum ({\apl f}) reduction in the inner loop is also stride-1
and vector-vector, offering similar performance benefits.
Fifth, the latter two vector operations can be loop-fused.
Finally, in the Boolean domain, an entire word from {\apl X} can
be quickly checked to see if it is all zeros or all ones.
This often permits even more work to be elided.

The author realized that the STAR algorithm should work extremely well
with APL's Boolean arrays, for the above reasons, and because
it could perform the inner loop a word or a matrix row at a time,
producing, by itself, a substantial speedup.
David Allen, then working in the APL Development Group, AKA ``The Zoo",
at I.P. Sharp Associates Limited, implemented our algorithm in 
the SHARP APL interpreter. The computational complexity 
of inner product allowed us to obtain
final performance improvements on Boolean arrays of
about three orders of magnitude (32 times 32). 

The STAR APL algorithm offers other benefits.
Looking at the STAR APL inner loop, we see that it computes a
scalar-vector function on {\apl g}:

\medskip
{\apl Xel \qtimes~Y\qlbr\0J\qsemic\qrbr}\\

\noindent We exploited some
simple algebra to eliminate some or all of the work done
in each of these computations.~\cite{RBernecky:apex}
Consider {\apl 0 1 \qor\qdot\qand~Y}: In the first iteration,
{\apl Xel} is {\apl 0}, {\apl f} is {\apl \qor}, and {\apl g} is {\apl \qand}.
Hence, the {\apl g} computation is:

{\apl 0~\qand~Y\qlbr0\qsemic\qrbr}

\noindent This will always generate a vector of zeros of the same
length as a row of {\apl Y}. Knowing this lets us skip
the {\apl g} computation for that element.
Furthermore, since zero is the identity element for {\apl \qor},
we can also skip the reduction part of the computation on {\apl f}. 

The author, while adapting the STAR APL algorithm for use
in the SAC back end of the APEX APL compiler, developed 
a further performance enhancement, along these same 
lines.\cite{RBernecky:apex}
Some time later, Hui independently rediscovered both of these
enhancements.\cite{RKWHui:innerproduct}
The STAR APL and STARLIKE algorithms are now used in 
the APEX APL compiler, Dyalog APL, the J interpreter, and
in Walter Fil's Plural compiler.

That enhancement, which we dubbed {\em STARLIKE},
recognized, essentially, that
in the second iteration, {\apl Xel}
is {\apl 1}, which is the identity element for {\apl \qand}.
This lets us, again, skip the {\apl g}-computation entirely,
and then perform the {\apl f}-reduction using the current row of Y.

An added advantage of the STAR algorithm is that it works well
on all data types, providing a crude form of sparse array support
for elements of {\apl X}.

